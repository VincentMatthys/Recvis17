\documentclass{article}
\usepackage[utf8]{inputenc}
\usepackage[T1]{fontenc}
\usepackage[french]{babel}

% ------------------------- Color table ----------------------------------------
\usepackage{multirow}
\usepackage[table]{xcolor}
\definecolor{maroon}{cmyk}{0,0.87,0.68,0.32}
% ------------------------------------------------------------------------------

\usepackage{amscd}
\usepackage{amsthm}
\usepackage{physics}
\usepackage[left=2.2cm,right=2.2cm,top=2cm,bottom=2cm]{geometry}
\usepackage{textcomp,gensymb} %pour le °C, et textcomp pour éviter les warning
\usepackage{graphicx} %pour les images
\usepackage{caption}
\usepackage{subcaption}
\usepackage[colorlinks=true,
	breaklinks=true,
	citecolor=blue,
	linkcolor=blue,
	urlcolor=blue]{hyperref} % pour insérer des liens
\usepackage{epstopdf} %converting to PDF
\usepackage[export]{adjustbox} %for large figures

\usepackage{array}
\usepackage{dsfont}% indicatrice : \mathds{1}


% -------------------------- Mathematics ---------------------------------------
\graphicspath{{images/}{../images/}} % For the images path
% ------------------------------------------------------------------------------

% -------------------------- Mathematics ---------------------------------------
\usepackage{mathrsfs, amsmath, amsfonts, amssymb}
\usepackage{bm}
\usepackage{mathtools}
\usepackage[Symbol]{upgreek} % For pi \uppi different from /pi
\newcommand{\R}{\mathbb{R}} % For Real space

% ------------------------------------------------------------------------------


% -------------------------- Code format ---------------------------------------
\usepackage[numbered,framed]{matlab-prettifier}
\lstset{
	style              = Matlab-editor,
	basicstyle         = \mlttfamily,
	escapechar         = '',
	mlshowsectionrules = true,
}
% ------------------------------------------------------------------------------

% ------------------------- Blbiographie --------------------------------------
\usepackage[backend=biber, style=ieee]{biblatex}
\addbibresource{biblio.bib}
% ------------------------------------------------------------------------------


\setcounter{tocdepth}{4} %Count paragraph
\setcounter{secnumdepth}{4} %Count paragraph
\usepackage{float}

\usepackage{graphicx} % for graphicspath
% \graphicspath{{../images/}}

\usepackage{array,tabularx}
\newcolumntype{L}[1]{>{\raggedright\let\newline\\\arraybackslash\hspace{0pt}}m{#1}}
\newcolumntype{C}[1]{>{\centering\let\newline\\\arraybackslash\hspace{0pt}}m{#1}}
\newcolumntype{R}[1]{>{\raggedleft\let\newline\\\arraybackslash\hspace{0pt}}m{#1}}

% to start counting section to 6


%%%%%%%%%%%%%%%%%%%%%%%%%%%%%%%%%%%%%%%%%%%
% Header
%%%%%%%%%%%%%%%%%%%%%%%%%%%%%%%%%%%%%%%%%%%

\renewcommand{\assignmenttitle}{Assignment 1: Title of the assignment}
\renewcommand{\studentname}{Vincent Matthys}
\renewcommand{\email}{vincent.matthys@ens-paris-saclay.fr}

%%%%%%%%%%%%%%%%%%%%%%%%%%%%%%%%%%%%%%%%%%%
% Syntax for using figure macros:
%%%%%%%%%%%%%%%%%%%%%%%%%%%%%%%%%%%%%%%%%%%

% \singlefig{filename}{scalefactor}{caption}{label}
% \doublefig{\subfig{filename}{scalefactor}{subcaption}{sublabel}}
%           {\subfig{filename}{scalefactor}{subcaption}{sublabel}}
%           {global caption}{label}
% \triplefig{\subfig{filename}{scalefactor}{subcaption}{sublabel}}
%           {\subfig{filename}{scalefactor}{subcaption}{sublabel}}
%           {\subfig{filename}{scalefactor}{subcaption}{sublabel}}
%           {global caption}{label}
%
% Tips:
% - with scalefactor=1, a single figure will take the whole page width; a double figure, half page width; and a triple figure, a third of the page width
% - image files should be placed in the image folder
% - no need to put image extension to include the image
% - for vector graphics (plots), pdf figures are suggested
% - for images, jpg/png are suggested
% - labels can be left empty {}

%%%%%%%%%%%%%%%%%%%%%%%%%%%%%%%%%%%%%%%%%%%
% Beginning of assignment
%%%%%%%%%%%%%%%%%%%%%%%%%%%%%%%%%%%%%%%%%%%
\begin{document}
\maketitle

\question{QIA.1: (i) Why is it important to have a similarity co-variant feature detector? (ii) How does this affect the descriptors computed at these detections? (iii) How does this affect the matching process?}

\begin{enumerate}
	\item It is important to have a similarity co-variant feature detector in order to extract the same regions of interest regardless of translations, rotations or scales, \textit{i.e.} regardless of viewpoints.
	\item The fact that the feature detector is co-variant causes the descriptors computed at these detections to undergo the same tranformations than the image.
	\item Having feature detectors with similarity co-variance and feature descriptors with this and that other property affects the matching process in this way.
\end{enumerate}

\question{QIA.2: Show the detected features in the two images for three different values of the peakThreshold option}

\triplefig{\subfig{IA_2_1}{1}{\(peakThreshold = 0.0001\)}{fig:002:th_0.0001}} % sublabels used here
{\subfig{IA_2_2}{1}{\(peakThreshold = 0.001\)}{fig:002:th_0.001}}
{\subfig{IA_2_3}{1}{\(peakThreshold = 0.01\)}{fig:002:th_0.01}}
{Feature detectors with three different values of \(peakThreshold\) for all\_souls\_000002.jpg}{fig:002}

\triplefig{\subfig{IA_2_4}{1}{\(peakThreshold = 0.0001\)}{fig:015:th_0.0001}} % sublabels used here
{\subfig{IA_2_5}{1}{\(peakThreshold = 0.001\)}{fig::015:th_0.001}}
{\subfig{IA_2_6}{1}{\(peakThreshold = 0.01\)}{fig::015:th_0.01}}
{Feature detectors with three different values of \(peakThreshold\) for all\_souls\_000015.jpg}{fig:015}

The detected features for \(peakThreshold \in \{0.0001, 0.001, 0.01\}\) are shown in Figure~\ref{fig:002} for the all\_souls\_000002.jpg image, and in Figure~\ref{fig:015} for the all\_souls\_000015.jpg image. As expected, the number of keypoints decreses with the threshold. Moreover the keypoints are fewer in the darker regions, wich leads to asymetric density in the Figure~\ref{fig:015}, especially because of the building shadow.

\question{QIA.3: Note the change in spatial density of detections across images, for a given value of peakThreshold. (i) Is the density uniform? If not, why? (ii) Which implications for image matching can this have? (iii) How can it be avoided?}
\begin{enumerate}
	\item In Figures~\ref{fig:002:th_0.0001} and~\ref{fig:015:th_0.0001}. The density is clearly not uniform between the two images. In both, the top-center of the image is very dense, but, in the first one, the grass admits no detection, which leads to a gap in the density, with some detections in the bottom limit of the grass.
	\item This can lead to error matching the correspondances with local features
	\item To avoid it we can proceed to a spatial verification relying on global geometric relations.
\end{enumerate}




\end{document}
