\documentclass{article}
\usepackage[utf8]{inputenc}
\usepackage[T1]{fontenc}
\usepackage[french]{babel}

% ------------------------- Color table ----------------------------------------
\usepackage{multirow}
\usepackage[table]{xcolor}
\definecolor{maroon}{cmyk}{0,0.87,0.68,0.32}
% ------------------------------------------------------------------------------

\usepackage{amscd}
\usepackage{amsthm}
\usepackage{physics}
\usepackage[left=2.2cm,right=2.2cm,top=2cm,bottom=2cm]{geometry}
\usepackage{textcomp,gensymb} %pour le °C, et textcomp pour éviter les warning
\usepackage{graphicx} %pour les images
\usepackage{caption}
\usepackage{subcaption}
\usepackage[colorlinks=true,
	breaklinks=true,
	citecolor=blue,
	linkcolor=blue,
	urlcolor=blue]{hyperref} % pour insérer des liens
\usepackage{epstopdf} %converting to PDF
\usepackage[export]{adjustbox} %for large figures

\usepackage{array}
\usepackage{dsfont}% indicatrice : \mathds{1}


% -------------------------- Mathematics ---------------------------------------
\graphicspath{{images/}{../images/}} % For the images path
% ------------------------------------------------------------------------------

% -------------------------- Mathematics ---------------------------------------
\usepackage{mathrsfs, amsmath, amsfonts, amssymb}
\usepackage{bm}
\usepackage{mathtools}
\usepackage[Symbol]{upgreek} % For pi \uppi different from /pi
\newcommand{\R}{\mathbb{R}} % For Real space

% ------------------------------------------------------------------------------


% -------------------------- Code format ---------------------------------------
\usepackage[numbered,framed]{matlab-prettifier}
\lstset{
	style              = Matlab-editor,
	basicstyle         = \mlttfamily,
	escapechar         = '',
	mlshowsectionrules = true,
}
% ------------------------------------------------------------------------------

% ------------------------- Blbiographie --------------------------------------
\usepackage[backend=biber, style=ieee]{biblatex}
\addbibresource{biblio.bib}
% ------------------------------------------------------------------------------


\setcounter{tocdepth}{4} %Count paragraph
\setcounter{secnumdepth}{4} %Count paragraph
\usepackage{float}

\usepackage{graphicx} % for graphicspath
% \graphicspath{{../images/}}

\usepackage{array,tabularx}
\newcolumntype{L}[1]{>{\raggedright\let\newline\\\arraybackslash\hspace{0pt}}m{#1}}
\newcolumntype{C}[1]{>{\centering\let\newline\\\arraybackslash\hspace{0pt}}m{#1}}
\newcolumntype{R}[1]{>{\raggedleft\let\newline\\\arraybackslash\hspace{0pt}}m{#1}}

% to start counting section to 6


%%%%%%%%%%%%%%%%%%%%%%%%%%%%%%%%%%%%%%%%%%%
% Header
%%%%%%%%%%%%%%%%%%%%%%%%%%%%%%%%%%%%%%%%%%%

\renewcommand{\assignmenttitle}{Assignment 1: Title of the assignment}
\renewcommand{\studentname}{Vincent Matthys}
\renewcommand{\email}{vincent.matthys@ens-paris-saclay.fr}

%%%%%%%%%%%%%%%%%%%%%%%%%%%%%%%%%%%%%%%%%%%
% Syntax for using figure macros:
%%%%%%%%%%%%%%%%%%%%%%%%%%%%%%%%%%%%%%%%%%%

% \singlefig{filename}{scalefactor}{caption}{label}
% \doublefig{\subfig{filename}{scalefactor}{subcaption}{sublabel}}
%           {\subfig{filename}{scalefactor}{subcaption}{sublabel}}
%           {global caption}{label}
% \triplefig{\subfig{filename}{scalefactor}{subcaption}{sublabel}}
%           {\subfig{filename}{scalefactor}{subcaption}{sublabel}}
%           {\subfig{filename}{scalefactor}{subcaption}{sublabel}}
%           {global caption}{label}
%
% Tips:
% - with scalefactor=1, a single figure will take the whole page width; a double figure, half page width; and a triple figure, a third of the page width
% - image files should be placed in the image folder
% - no need to put image extension to include the image
% - for vector graphics (plots), pdf figures are suggested
% - for images, jpg/png are suggested
% - labels can be left empty {}

%%%%%%%%%%%%%%%%%%%%%%%%%%%%%%%%%%%%%%%%%%%
% Beginning of assignment
%%%%%%%%%%%%%%%%%%%%%%%%%%%%%%%%%%%%%%%%%%%
\begin{document}
\maketitle

\question{QIA.1: (i) Why is it important to have a similarity co-variant feature detector? (ii) How does this affect the descriptors computed at these detections? (iii) How does this affect the matching process?}

\begin{enumerate}
	\item It is important to have a similarity co-variant feature detector in order to extract the same regions of interest regardless of translations, rotations or scales, \textit{i.e.} regardless of viewpoints.
	\item The fact that the feature detector is co-variant causes the descriptors computed at these detections to undergo the same tranformations than the image.
	\item Having feature detectors with similarity co-variance and feature descriptors with this and that other property affects the matching process in this way.
\end{enumerate}

\question{QIA.2: Show the detected features in the two images for three different values of the peakThreshold option}

\triplefig{\subfig{IA_2_1}{1}{\(peakThreshold = 0.0001\)}{fig:002:th_0.0001}} % sublabels used here
{\subfig{IA_2_2}{1}{\(peakThreshold = 0.001\)}{fig:002:th_0.001}}
{\subfig{IA_2_3}{1}{\(peakThreshold = 0.01\)}{fig:002:th_0.01}}
{Feature detectors with three different values of \(peakThreshold\) for all\_souls\_000002.jpg}{fig:002}

\triplefig{\subfig{IA_2_4}{1}{\(peakThreshold = 0.0001\)}{fig:015:th_0.0001}} % sublabels used here
{\subfig{IA_2_5}{1}{\(peakThreshold = 0.001\)}{fig::015:th_0.001}}
{\subfig{IA_2_6}{1}{\(peakThreshold = 0.01\)}{fig::015:th_0.01}}
{Feature detectors with three different values of \(peakThreshold\) for all\_souls\_000015.jpg}{fig:015}

The detected features for \(peakThreshold \in \{0.0001, 0.001, 0.01\}\) are shown in Figure~\ref{fig:002} for the all\_souls\_000002.jpg image, and in Figure~\ref{fig:015} for the all\_souls\_000015.jpg image. As expected, the number of keypoints decreses with the threshold. Moreover the keypoints are fewer in the darker regions, wich leads to asymetric density in the Figure~\ref{fig:015}, especially because of the building shadow.

\question{QIA.3: Note the change in spatial density of detections across images, for a given value of peakThreshold. (i) Is the density uniform? If not, why? (ii) Which implications for image matching can this have? (iii) How can it be avoided?}
\begin{enumerate}
	\item In Figures~\ref{fig:002:th_0.0001} and~\ref{fig:015:th_0.0001}. The density is clearly not uniform between the two images. In both, the top-center of the image is very dense, but, in the first one, the grass admits no detection, which leads to a gap in the density, with some detections in the bottom limit of the grass.
	\item This can lead to error matching the correspondances with local features
	\item To avoid it we can proceed to a spatial verification relying on global geometric relations.
\end{enumerate}


\question{QIB.1: Note the descriptors are computed over a much larger region (shown in blue) than the detection (shown in green). Why is this a good strategy?}

\doublefig{\subfig{IB_2}{1}{4x4 grid arround the keypoint}{fig:4x4grid}}
{\subfig{IB_3_1}{1}{first NN match}{fig:firstNN}}
{Computation of SIFT descriptors and matching}{fig:IB1}


The larger regions are shown in Figure~\ref{fig:4x4grid}. It is a good strategy to be insensitive to small geometric deformations, and small variations of illumination, by taking the average over a larger region


\question{QIB.2: Examine carefully the mismatches and try to understand the different causes of mismatch. (i) In your report, present at least 3 of them and explain why the mistakes are being made. For example, is the change in lighting a problem? (ii) What additional constraints can be applied to remove the mismatches?}

\singlefig{IB_3}{1}{Three mismatches with SIFT features descriptors, subgroup of matches extracted from~\ref{fig:firstNN}}{fig:3miss}
\begin{enumerate}
	\item Three types of mismatch are presentend in Figure~\ref{fig:3miss}, and are extracted from Figure~\ref{fig:firstNN} \begin{itemize}
		      \item repetitive pattern (in blue)
		      \item constant local change in lighting leading to similarity between two different keypoints (in red)
		      \item ambiguity between a keypoint his first two nearest neighbors (in yellow)
	      \end{itemize}
	      A global change in lighting is irrelevant for matching, because of the gradient nullifying the constant component
	\item Geometric verification with global constraints can discard those mismatches. Another method, presented below, is to look at the second nearest neigbour to remove ambiguous matches.
\end{enumerate}

\question{QIC.1: Illustrate and comment on improvements that you can achieve with this step. Include in your report figures showing the varying number of tentative matches when you change the nnThreshold parameter.}

As expected, the percentage of matches kept increases with \( nnThreshold\). The results obtained are shown in Figures~\ref{fig:IC_12}~\ref{fig:IC_34}~\ref{fig:IC_5}. Until \( nnThreshold = 0.7\), there is no substantial mismatches, but the number of matches kept is below \( 10~\%\). If more matches are kept, a non-negligeable part of new matches is missmatches, because the ambiguity between the first and the second nearest neighbour is no longer removed. In Figure~\ref{fig:graphnnth} we can see the exponential growth of the number of matches with \( nnThreshold \). A good compromise would be to take the threshold before this exponential growth.

\doublefig{\subfig{IC_1}{1}{\( nnThreshold = 0.3\)}{fig:nth0.3}}
{\subfig{IC_2}{1}{\( nnThreshold = 0.5\)}{fig:nth0.5}}
{Matching using Lowe's second nearest neighbour - part1}{fig:IC_12}
\doublefig{\subfig{IC_3}{1}{\( nnThreshold = 0.7\)}{fig:nth0.7}}
{\subfig{IC_4}{1}{\( nnThreshold = 0.9\)}{fig:nth0.9}}
{Matching using Lowe's second nearest neighbour - part2}{fig:IC_34}
\doublefig{\subfig{IC_5}{1}{\( nnThreshold = 0.8\)}{fig:nth0.8}}
{\subfig{2nn_graph}{1}{Percentage of matches in function of the \( nnThreshold\)}{fig:graphnnth}}
{Matching using Lowe's second nearest neighbour - part3}{fig:IC_5}


\question{QID.1: Work out how to compute this transformation from a single correspondence. Hint: recall from Stage I.A that a SIFT feature frame is an oriented circle and map one onto the other.}


\question{QID.2: Illustrate improvements that you can achieve with this step.}
\doublefig{\subfig{ID_2}{1}{\( nnThreshold = 0.8\)}{fig:nth0.8_geo}}
{\subfig{geo_verif}{1}{Percentage of matches in function of the \( nnThreshold\)}{fig:graph_geo}}
{Improving SIFT matching with Lowe's 2nd nn and geometric verification}{fig:ID_2}
In Figure~\ref{fig:nth0.8_geo}, it's clear there is no mismatch anymore: every match corresponds to a unique geometric tranformation. The number of matches dropped from \( 293\) to \( 123\). It means only \( 42\%\) of the matches given by the Lowe's method were inliers. In Figure~\ref{fig:graph_geo} we can observe the almost linear growth of matches with \( nnThreshold\) with the geometric verification, in comparaison with the exponential growth of the Lowe's method alone.

\question{QII.1: Include in your report images and a graph showing the number of verified matches with changing viewpoint. At first there are more similarity detector matches than affine. Why?}
At first, introducing affine detectors is introducing a second spatial dimension for each blob, which leads to one more constraint to verify in the geometric transformation. This results discarding some outliers at the very beginning.

\question{QIIIA.1: In the above procedure the time required to convert the descriptors into visual words was not accounted for. Why the speed of this step is less important in practice?}

\end{document}
