\documentclass[12pt,a4paper,onecolumn]{article}
\input{packages}
\input{macros}

% ------------------------ General informations --------------------------------
\title{MVA - Object recognition - Project proposal}
\author{Vincent Matthys}
\graphicspath{{images/}{../images/}} % For the images path
% ------------------------------------------------------------------------------


\begin{document}

\begin{tabularx}{0.9\textwidth}{@{} l X r @{} }
	{\textsc{Master MVA}}       &  & \textsc{Vincent Matthys} \\
	\textsc{Object recognition} &  & {ENS Paris Saclay}       \\
\end{tabularx}
\vspace{1.5cm}
\begin{center}

	\rule[11pt]{5cm}{0.5pt}

	\textbf{\LARGE \textsc{Automatic estimatin of 3D pose from a signle image}}
	\vspace{0.5cm}

	Vincent Matthys

	vincent.matthys@ens-paris-saclay.fr

	\rule{5cm}{0.5pt}

	\vspace{1.5cm}
\end{center}

I decided to work on my one on project D named : \textbf{Automatic estimatin of 3D pose from a signle image}.
The plan of the work is the following :
\begin{itemize}
	\item Obtain 2d detections using stacked hourglass network of Newell et al. \cite{newell2016} pre-trained on MPII dataset. Fine tune on Human3.6M dataset using the parameters of Martinez et al \cite{martinez2017simple}.
	\item Implement the mapping between 2d points and 3d points as decribed in \cite{martinez2017simple}.
	\item Test the imeplentation of the Human3D dataset and characterize the working conditions and failures modes
	\item Extend to video by fixing the shape and optimizing the pose
	\item Test on the shorter manipulations videos dataset, and characterize the working conditions and failures modes
\end{itemize}

\printbibliography

\end{document}
