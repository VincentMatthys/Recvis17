\documentclass{article}
\usepackage[utf8]{inputenc}
\usepackage[T1]{fontenc}
\usepackage[french]{babel}

% ------------------------- Color table ----------------------------------------
\usepackage{multirow}
\usepackage[table]{xcolor}
\definecolor{maroon}{cmyk}{0,0.87,0.68,0.32}
% ------------------------------------------------------------------------------

\usepackage{amscd}
\usepackage{amsthm}
\usepackage{physics}
\usepackage[left=2.2cm,right=2.2cm,top=2cm,bottom=2cm]{geometry}
\usepackage{textcomp,gensymb} %pour le °C, et textcomp pour éviter les warning
\usepackage{graphicx} %pour les images
\usepackage{caption}
\usepackage{subcaption}
\usepackage[colorlinks=true,
	breaklinks=true,
	citecolor=blue,
	linkcolor=blue,
	urlcolor=blue]{hyperref} % pour insérer des liens
\usepackage{epstopdf} %converting to PDF
\usepackage[export]{adjustbox} %for large figures

\usepackage{array}
\usepackage{dsfont}% indicatrice : \mathds{1}


% -------------------------- Mathematics ---------------------------------------
\graphicspath{{images/}{../images/}} % For the images path
% ------------------------------------------------------------------------------

% -------------------------- Mathematics ---------------------------------------
\usepackage{mathrsfs, amsmath, amsfonts, amssymb}
\usepackage{bm}
\usepackage{mathtools}
\usepackage[Symbol]{upgreek} % For pi \uppi different from /pi
\newcommand{\R}{\mathbb{R}} % For Real space

% ------------------------------------------------------------------------------


% -------------------------- Code format ---------------------------------------
\usepackage[numbered,framed]{matlab-prettifier}
\lstset{
	style              = Matlab-editor,
	basicstyle         = \mlttfamily,
	escapechar         = '',
	mlshowsectionrules = true,
}
% ------------------------------------------------------------------------------

% ------------------------- Blbiographie --------------------------------------
\usepackage[backend=biber, style=ieee]{biblatex}
\addbibresource{biblio.bib}
% ------------------------------------------------------------------------------


\setcounter{tocdepth}{4} %Count paragraph
\setcounter{secnumdepth}{4} %Count paragraph
\usepackage{float}

\usepackage{graphicx} % for graphicspath
% \graphicspath{{../images/}}

\usepackage{array,tabularx}
\newcolumntype{L}[1]{>{\raggedright\let\newline\\\arraybackslash\hspace{0pt}}m{#1}}
\newcolumntype{C}[1]{>{\centering\let\newline\\\arraybackslash\hspace{0pt}}m{#1}}
\newcolumntype{R}[1]{>{\raggedleft\let\newline\\\arraybackslash\hspace{0pt}}m{#1}}

% to start counting section to 6


%%%%%%%%%%%%%%%%%%%%%%%%%%%%%%%%%%%%%%%%%%%
% Header
%%%%%%%%%%%%%%%%%%%%%%%%%%%%%%%%%%%%%%%%%%%

\renewcommand{\assignmenttitle}{Assignment 2 :Image classification}
\renewcommand{\studentname}{Vincent Matthys}
\renewcommand{\email}{vincent.matthys@ens-paris-saclay.fr}

% renew (sub)section to get alpanumeric characters
\renewcommand{\thesection}{\Alph{section}}

%%%%%%%%%%%%%%%%%%%%%%%%%%%%%%%%%%%%%%%%%%%
% Syntax for using figure macros:
%%%%%%%%%%%%%%%%%%%%%%%%%%%%%%%%%%%%%%%%%%%

% \singlefig{filename}{scalefactor}{caption}{label}
% \doublefig{\subfig{filename}{scalefactor}{subcaption}{sublabel}}
%           {\subfig{filename}{scalefactor}{subcaption}{sublabel}}
%           {global caption}{label}
% \triplefig{\subfig{filename}{scalefactor}{subcaption}{sublabel}}
%           {\subfig{filename}{scalefactor}{subcaption}{sublabel}}
%           {\subfig{filename}{scalefactor}{subcaption}{sublabel}}
%           {global caption}{label}
%
% Tips:
% - with scalefactor=1, a single figure will take the whole page width; a double figure, half page width; and a triple figure, a third of the page width
% - image files should be placed in the image folder
% - no need to put image extension to include the image
% - for vector graphics (plots), pdf figures are suggested
% - for images, jpg/png are suggested
% - labels can be left empty {}

%%%%%%%%%%%%%%%%%%%%%%%%%%%%%%%%%%%%%%%%%%%
% Beginning of assignment
%%%%%%%%%%%%%%%%%%%%%%%%%%%%%%%%%%%%%%%%%%%
\begin{document}
\maketitle

\part{}

\section{Stage A}


\question{QIA.1: (i) Why is it important to have a similarity co-variant feature detector? (ii) How does this affect the descriptors computed at these detections? (iii) How does this affect the matching process?}

\begin{enumerate}
	\item It is important to have a similarity co-variant feature detector because of this reason. It is also important because of this other reason.
	\item The fact that the feature detector is co-variant causes the descriptors computed at these detections to have this and this other property.
	\item Having feature detectors with similarity co-variance and feature descriptors with this and that other property affects the matching process in this way.
\end{enumerate}

\question{QIA.2: Show the classification boundaries learned with the two different models. Compare their training error, and the expected test error, and discuss which model is superior.}

The two classification boundaries are shown in Figure~\ref{fig:overfit}. As it can be observed, model A, represented in green, is producing a very irregular classification boundary which gives a training error of zero, while model B, represented in black, is producing a smoother classification boundary which mis-classifies a few examples of the training data.

However, although the training error is higher for model B than for model A, we expect model B to perform better on the test data, as its smoother classification boundary is more robust to the noise in the training data, while model A is overfitting to the noise in this particular training data.

It is therefore concluded that model B is superior, as it will offer better generalization properties.

\singlefig{overfitting_vectorial}{0.45}{Two different models for fitting the data. The classification boundaries for models A and B are shown in green and black respectively.}{fig:overfit}

\question{QIA.3: Show the detected features in the image for three different values of the peakThreshold option.}

The detected features with three different values of $peakThreshold$ are shown in Figure~\ref{fig:peakThreshold}. % if we want to refer to one particular subplot, we can use its sublabel instead: e.g. \ref{fig:th_0.0001}

\triplefig{\subfig{features_th1}{1}{$peakThreshold = 0.0001$}{fig:th_0.0001}} % sublabels used here
{\subfig{features_th2}{1}{$peakThreshold = 0.001$}{fig:th_0.001}}
{\subfig{features_th3}{1}{$peakThreshold = 0.01$}{fig:th_0.01}}
{Feature detectors with three different values of $peakThreshold$}{fig:peakThreshold}

\end{document}
